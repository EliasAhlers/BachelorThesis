\chapter{Conclusion}
\label{ch:conclusion}

In conclusion, this thesis has demonstrated the significant potential of LLMs in classifying and transferring between German language proficiency levels. Through exploration and application of prompt engineering and fine-tuning techniques, I have shown that LLMs can effectively tackle the complex task of CEFR classification, achieving results that surpass previous approaches.

The fine-tuned LLaMA-3-8B-Instruct model achieved an impressive 77.3\% accuracy across all six CEFR levels, with a 100\% group accuracy on the test set. This performance shows the model's ability to make small distinctions between proficiency levels while avoiding drastic misclassifications. The transfer task, while more challenging, also showed promising results, with the fine-tuned model nearly doubling the transfer accuracy to 34.9\% compared to the baseline. But here is still room for improvement for further research.

Comparing my findings to previous studies reveals that the LLM-based approach not only matches but exceeds the performance of traditional machine learning methods. This is especially interesting given the finer distinctions between six CEFR levels in this study.

However, limitations such as the relatively small and imbalanced dataset suggest room for improvement. Future work could focus on expanding the dataset with authentic learner texts, developing more sophisticated evaluation metrics for the transfer task, exploring continuous scoring within CEFR levels, and investigating the application of these models in real-world language learning environments.

In summary, this thesis has demonstrated the effectiveness of LLMs, especially in German CEFR classification tasks. There is great potential for further research, especially regarding the transfer task.

% The conclusion chapter should summarize the key findings of your study and draw out implications and contributions of your work. It should connect the different elements of your research, reinforcing the importance of the results and their impact on the field in general. In particular, the conclusion should reconnect with the introduction and the original research question and should evaluate if the goals of the work have been reached.

% If applicable, outline practical implications of the findings: How can this research be applied in real-world scenarios? But also, acknowledge any limitations of the study and suggest how this might be addressed in future work. Subsequent research questions should be clearly identified and follow-up studies pointed out.

% Last, the thesis should end by summarizing key insights and stressing the importance of these.

% \section{Tips for the Conclusion}
% \begin{itemize}
%     \item Recap the main results and address the research questions. This should connect the findings back to the objectives stated in the introduction.
%     \item Highlight the contributions of the research --- acknowledge the limitations of the study.
%     \item Discuss the practical implications of the findings --- suggest directions for future research.
%     \item Avoid introducing new information.
%     \item Keep the conclusion concise and focused.
%     \item Discussion and Conclusion are often combined into one section (when these are not too long).
% \end{itemize}